\documentclass{sig-alternate}

\usepackage{graphicx}

\begin{document}

\conferenceinfo{Models@Run.Time}{'12 Innsbruck, Austria}

\title{Model-Driven Generation of DSML Execution Engines\titlenote{This work was partly supported by the Capes Foundation, Brazil, Proc. 0759-11-2}}

\numberofauthors{2}

\author{
\alignauthor Gustavo C. M. Sousa \\ F{\'a}bio M. Costa\\
       \affaddr{Instituto de Inform{\'a}tica}\\
       \affaddr{Universidade Federal de Goi{\'a}s}\\
       \affaddr{Goi{\^a}nia-GO, Brazil}\\
       \email{\{gustavo|fmc\}@inf.ufg.br}
%\alignauthor F{\'a}bio M. Costa\\
%       \affaddr{Instituto de Inform{\'a}tica}\\
%       \affaddr{Universidade Federal de Goi{\'a}s}\\
%       \affaddr{Goi{\^a}nia-GO, Brazil}\\
%       \email{fmc@inf.ufg.br}
\alignauthor Peter J. Clarke\\
       \affaddr{School of Computing and\\ Information Sciences}\\
       \affaddr{Florida International University}\\
       \affaddr{Miami-FL, USA}\\
       \email{clarkep@cis.fiu.edu}
}

\date{July 2012}

\maketitle

\begin{abstract}
The combination of domain-specific modeling languages and model-driven engineering techniques hold the promise of a breakthrough in the way applications are developed. 
By raising the level of abstraction and specializing in building blocks that are familiar in a particular domain, it has the potential to turn domain experts into application developers. 
Applications are developed as models, which in turn are interpreted at runtime by a specialized execution engine in order to produce the intended behavior. 
% are interpreted or may be interpreted?
This approach has been successfully applied in different domains, such as communication and smart grid management. 
However, each time the approach has to be realized in a different domain, substantial re-implementation has to take place in order to put together an execution engine for the respective DSML. 
In this paper, we present our work towards a generalization of the approach in the form of a meta-model and its respective execution environment, which capture the domain-independent aspects of runtime model interpretation and allow the definition of domain-specific execution engines as instances of the meta-model. 
We present an initial validation of the approach in the context of the Communication Virtual Machine project, by realizing part of the execution engine architecture in the form of an instance of the proposed meta-model. 
\end{abstract}

\category{D.2.11} {Software Engineering}{Software Architectures}[domain-specific architectures, languages]

\keywords{Models at Runtime, Model-Driven Engineering, Domain-Specific Modeling Languages, Metamodeling, Middleware}

\section{Introduction}
%TODO: FOR THE WHOLE SECTION... TRY TO FOCUS MORE ON THE MODELS@RUNTIME. I MEAN, I NEED TO MAKE MORE EXPLICIT THE RELATIONSHIP BETWEEN THE WORK AND MODELS@RUNTIME

% Tell the whole story: problem, proposed approach to solution, results. 
Model driven engineering is being proposed as an approach to deal with the growing complexity involved in the development of modern applications.
These applications commonly operate in a distributed and heterogeneous environment and must behave in a dependable manner. 
These applications also often need to adapt at runtime to comply with changing requirements and environment.
In order to achieve these goals the developers need to employ a variety of strategies to 

MDE proposes to simplify the development of applications.
By employing models built upon abstractions of the problem domain.
MDE looks forward reducing the complexity due to the the technical details in the translation from domain concepts to platform concepts.

% OBS: First we introduce the MDE approach of using DSMLs and automated processing
In order to reduce this complexity, MDE relies on DSMLs and automated processing of models.
Models can be either transformed or interpreted.

% OBS: Then we discuss about the DSML
The use of high level (and maybe graphical) DSMLs?
Making it possible for domain experts, or even end users to create applications from their requirements.

% OBS: And next we discuss about the automated processing as an interpreter (exeution engine)
Communication Virtual Machine (CVM) and MicroGrid Virtual Machine (MGridVM) are examples of software that employ MDE techniques to allow users to develop applications using DSMLs.
Don't forget to mention that its not only applications, but complex applications
Besides allowing the development of applications, these software also enables the users to adapt applications during their runtime by manipulating models. These virtual machines are execution engines for DSMLs named CML and MGridML respectively.
% TODO: make the concept of execution engine clearer
% TODO: MAKE IT CLEAR THAT THE MODELS PROCESSED BY THESE ARE MODELS@RUNTIME

% QUESTION: maybe change the order of ideas introducing the need to embody knowledge into the DSML and execution engine before the CVM and MGridVM
% QUESTION: maybe introduce it not in terms of CVM/MGridVM
% QUESTION: While mentioning CVM should we mention here that it works by analyzing changes in the model and synthesizing scripts? Or should we delay it to the Background?
%ALTERNATIVE: State that the use of an interpreter (herein called execution engine) enables the use of models to also change the runtime behavior using constructs closer to the problem domain

to allow the above (user built applications)?
DSMLs, along with the mechanisms for their processing embed concepts and semantics of the domain.
Once we have a DSML and the mechanism for processing it, users can directly develop applications in that DSML.

While the development of applications in these domains employing their respective DSMLs is quite simple, the same is not true when talking about the execution engine that interprets their languages. These mechanisms are commonly developed in general purpose language and are subject to all the technical complexities involved in the traditional software development.

Despite that, these execution engines share a lot of commonalities.
Their applications basically provide a high-level service upon a set of heterogeneous resources.
In order to do that, the runtime engine may need to execute many operations related to the processing of models, negotiation of configurations, synthesis of commands, evaluation of policies, runtime adaptation, and etc.
%TODO: Give a better look at how the chapter about the CVM mentions these features and the similarity between these VMs
Both CVM and MGridVM are built upon a layered architecture, in which each layer has well defined responsibilities.
%Although the communication domain is the focus of this chapter, we are currently applying the approach to other domains, such as the microgrid energy management applications (Allison, Allen, Yang, & Clarke, 2011). That is, we are in the process of developing the Microgrid Modeling Language (MGridML) and the Microgrid Virtual Machine (MGridVM) using the approach describe here. By applying our approach to multiple domains, we hope to eventually show how the approach can be generalized as a new way for defining semantics for i-DSMLs that can be directly executed by a virtual machine.

Taking that into consideration, we propose a model-driven approach to the construction of execution engines for the provision of high-level services described by DSMLs upon a set of resources. Doing so, we look forward a way to assist the construction of DSML execution engines. 
%Generalization of approach. The long-term goal of this research project is to generalize the approach on developing i-DSMLs. We expect to create an infrastructure that can be used to assist with developing the i-DSML (see Figure 1), capturing the domain-specific behavior (maybe using LTSs e.g., Tables 4 and 5), and generating the domain specific VM (transformation shown in Figure 4).
%Such an infrastructure would require: (1) defining a set of domain characteristics that are well suited for these i-DSMLs, (2) a meta-metamodel that can be used to define both the i-DSML and its interpreter, (3) a method for capturing the domain-specific behavior and integrating it into the interpreter.
Our approach relies on the experience acquired in the development of CVM and MGridVM to propose a generic architecture and the use of modeling as a way to specialize it.
We illustrate this approach by presenting a meta-model designed to describe one of the layers of the proposed architecture.


\section{Background}

% MDE, DSMLs and models at runtime; analogy with the distinction between structural reflection (modeling of the language) and behavioral reflection (modeling of the execution engine);

models@runtime s�o uma forma de autorepresenta��o com causalidade i.e. reflex�o mais pr�xima do dom�nio

\section{Generic Architecture of the Execution Engine}

% Describe the overall design without going into details of the meta-model.
% Structure the description around a block diagram. Diagrama de blocos do fluxo para empregar a abordagem

%Emphasize the separation between the domain-independent aspect of model execution and the definition of the DSML (and how you link a particular execution engine definition to its respective DSML).
Dada uma DSML previamente definida, podemos definir um execution engine que vai executar instancias dessa DSML
Futuras mudan�as no execution engine podem ser para atender mudan�as na linguagem ou para mudar requisitos n�o funcionais
liga��o entre linguagem e plataforma n�o � formalizado

%Discuss the general application of the approach in CVM-like architectures, but limit it to the broker layer for the purposes of scoping the current work.

%Justify the reduction of scope to the broker

\section{Meta-model for Broker Layer}

% Show the class diagram and describe elements of the meta-model. May structure the section based on the major blocks of the meta-model.

\subsection{Interface}
\subsection{Signal handling}
\subsection{Resource management}
\subsection{State management}
\subsection{Autonomic computing}
\subsection{Policies}

\section{Execution Environment}

% Describe the execution engine based on EMF. This section corresponds to the implementation section that goes in most papers. Make it clear in the text that it is the implementation.

\section{Example in the Communication Domain}

% Present the NCB instance (model) . Describe its use: how it is instantiated (i.e., put to run in the form of an execution engine) and how it interprets a user-derived model at runtime (describe the model that is seen by the NCB layer, i.e., based on the commands available at its interface with the UCM layer).

\section{Related Work}
Quais as �reas relacionadas, e trabalhos relacionados
- Outras abordagens de constru��o de execution engines de DSMLs

% Related work should be in the area of runtime model interpretation and execution engines for DSMLs. Must present them and compare with your approach.

\section{Concluding Remarks}

% Summarize the contributions and discuss limitations and future work.

\section{References}

\bibliographystyle{abbrv}
\bibliography{GustavoMRT2012}

\end{document}