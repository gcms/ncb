\keys{Model-Driven Engineering, Domain-Specific Modeling Languages, Metamodeling}

\begin{abstract}{Model-driven development of Domain-Specific Virtual Machines}
The combination of domain-specific modeling languages and model-driven engineering techniques hold the promise of a breakthrough in the way applications are developed. 
By raising the level of abstraction and specializing in building blocks that are familiar in a particular domain, it has the potential to turn domain experts into application developers. 
Applications are developed as models, which in turn are interpreted at runtime by a specialized execution engine in order to produce the intended behavior.
In this approach models are processed by domain-specific execution engines that embed knowledge about how to execute the models.
This approach has been successfully applied in different domains, such as communication and smart grid management to execute applications described by models that can be created and changed at runtime. 
However, each time the approach has to be realized in a different domain, substantial re-implementation has to take place in order to put together an execution engine for the respective DSML. 
In this work, we present a generalization of the approach in the form of a  metamodel that captures the domain-independent aspects of runtime model interpretation and allow the definition of a particular class of domain-specific execution engines which provide a high-level service upon an underlying set of heterogenous set of resources.
\end{abstract}
