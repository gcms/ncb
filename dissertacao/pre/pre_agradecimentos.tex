\begin{agradecimentos}


Antes de tudo, gostaria de agradecer de forma especial ao meu orientador e professor F�bio Moreira Costa por toda a aten��o e paci�ncia dedicada � minha orienta��o. 
Sem o seu cont�nuo apoio e motiva��o, este trabalho certamente n�o seria poss�vel.

Tamb�m quero agradecer ao Prof. Peter Clarke e o grupo de pesquisa da CVM na Florida International University n�o s� pelas discuss�es e v�rias contribui��es � este trabalho, mas tamb�m pela forma como me acolheram durante minha visita. Em especial, quero agradecer ao Dr. Andrew Allen pela sua disposi��o em ajudar no esclarecimento de minhas d�vidas.

Agrade�o tamb�m ao colega Roberto Vito Rodrigues Filho pela sua ajuda no in�cio deste trabalho e a todos os demais colegas que me acompanharam e me incentivaram durante esses anos.

Tamb�m devo agradecer � Coordena��o de Aperfei�oamento de Pessoal de N�vel Superior e � Funda��o de Amparo � Pesquisa do Estado de Goi�s pelo apoio financeiro para o desenvolvimento deste trabalho.
%, e � Universidade Federal de Goi�s que por meio de seus professores e servidores � respons�vel por toda minha forma��o acad�mica.

%\textless Texto com agradecimentos �quelas pessoas/entidades que, na opini�o do autor, deram alguma contribu���o relevante para o desenvolvimento do trabalho.\textgreater
\end{agradecimentos}


